\section{Conclusiones}
\begin{frame}
	Se logró compilar exitosamente el \textit{kernel} de \texttt{OpuntiaOS} para
	las arquitecturas \texttt{ARM} y \texttt{x86}, utilizando el código de su repositorio.
	
	\enter
	
	Se logró emular, con ayuda de la herramienta \texttt{QEMU}, el sistema
	\texttt{OpuntiaOS} bajo las arquitecturas \texttt{ARM} y \texttt{x86}.
	
	\enter
	
	Fue posible comprender el funcionamiento del \textit{bootloader} de \texttt{OpuntiaOS} para el correcto arranque del sistema para las arquitecturas \texttt{ARM} y \texttt{x86}.
	
	\enter
	
	Se compararon lo puntos principales del arranque de ambas arquitecturas, teniendo en cuenta los modos que ambos manejan para dar control al sistema operativo.
\end{frame}


\begin{frame}
	Se tuvieron dificultades al arrancar el sistema en máquinas reales, se tiene la emulación con ayuda de \texttt{QEMU}.
	
	\enter
	
	Se hizo una relación de las estructuras de datos, así como de la estructura de los archivos fuente del \textit{kernel} de \texttt{OpuntiaOS} con aquellos respectivos vistos en el curso.
	
	\enter
	
	Finalmente, pese a que no fue posible ver \texttt{OpuntiaOS} fincionando en una máquina real, se comprendió su funcionamiento.
	
\end{frame}