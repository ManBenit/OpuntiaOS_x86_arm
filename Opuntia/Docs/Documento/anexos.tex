\section{Anexos \label{sec:anexos}}




\subsection{Estructuras de datos del administrador de procesos}
\begin{lstlisting}[
	language=C,
	label=prog:struct_proc,
	caption={Estructura de datos que define un proceso.\\
	Archivo: \texttt{kernel/include/tasking/proc.h}.}
	]
	struct proc {
		vm_address_space_t* address_space;
		
		pid_t pid;
		pid_t ppid;
		pid_t pgid;
		uint32_t prio;
		uint32_t status;
		struct thread* main_thread;
		
		spinlock_t lock;
		
		uid_t uid;
		gid_t gid;
		uid_t euid;
		gid_t egid;
		uid_t suid;
		gid_t sgid;
		
		file_t* proc_file;
		path_t cwd;
		file_descriptor_t* fds;
		
		int exit_code;
		
		bool is_kthread;
		
		// Trace data
		bool is_tracee;
	};
	typedef struct proc proc_t;
\end{lstlisting}

\begin{lstlisting}[
	language=C,
	label=prog:struct_thread,
	caption={Estructura de datos \texttt{thread} para soportar hilos de ejecución de procesos.\\
	Archivo: \texttt{kernel/include/tasking/thread.h}.}
	]
	struct thread {
		struct proc* process;
		pid_t tid;
		uint32_t status;
		
		/* Kernel data */
		kmemzone_t kstack;
		context_t* context; // context of kernel's registers
		trapframe_t* tf;
		#ifdef FPU_ENABLED
		fpu_state_t* fpu_state;
		#endif
		
		/* Scheduler data */
		struct thread* sched_prev;
		struct thread* sched_next;
		int last_cpu;
		time_t ticks_until_preemption;
		time_t start_time_in_ticks; // Time when the task was put to run.
		
		/* Blocker data */
		blocker_t blocker;
		union {
			blocker_join_t join;
			blocker_rw_t rw;
			blocker_sleep_t sleep;
			blocker_select_t select;
		} blocker_data;
		
		/* Stat data */
		time_t stat_total_running_ticks;
		
		uint32_t signals_mask;
		uint32_t pending_signals_mask;
		void* signal_handlers[SIGNALS_CNT];
	};
	typedef struct thread thread_t;
	
	#define THREADS_PER_NODE (128)
	struct thread_list_node {
		thread_t thread_storage[THREADS_PER_NODE];
		struct thread_list_node* next;
		int empty_spots;
	};
	typedef struct thread_list_node thread_list_node_t;
	
	struct thread_list {
		struct thread_list_node* head;
		struct thread_list_node* next_empty_node;
		int next_empty_index;
		spinlock_t lock;
		struct thread_list_node* tail;
	};
	typedef struct thread_list thread_list_t;
\end{lstlisting}

\begin{lstlisting}[
	language=C,
		label=prog:struct_signals,
		caption={Definición de señales para los procesos.\\
		Archivo: \texttt{kernel/include/libkern/bits/signal.h}.}
	]
	#ifndef _KERNEL_LIBKERN_BITS_SIGNAL_H
	#define _KERNEL_LIBKERN_BITS_SIGNAL_H
	
	#define SIGHUP 1
	#define SIGINT 2
	#define SIGQUIT 3
	#define SIGILL 4
	#define SIGTRAP 5
	#define SIGABRT 6
	#define SIGIOT 6
	#define SIGBUS 7
	#define SIGFPE 8
	#define SIGKILL 9
	#define SIGUSR1 10
	#define SIGSEGV 11
	#define SIGUSR2 12
	#define SIGPIPE 13
	#define SIGALRM 14
	#define SIGTERM 15
	#define SIGSTKFLT 16
	#define SIGCHLD 17
	#define SIGCONT 18
	#define SIGSTOP 19
	#define SIGTSTP 20
	#define SIGTTIN 21
	#define SIGTTOU 22
	#define SIGURG 23
	#define SIGXCPU 24
	#define SIGXFSZ 25
	#define SIGVTALRM 26
	#define SIGPROF 27
	#define SIGWINCH 28
	#define SIGIO 29
	#define SIGPOLL SIGIO
	#define SIGPWR 30
	#define SIGSYS 31
	
	typedef void (*sighandler_t)(int);
	
	#define SIG_DFL ((sighandler_t)0)
	#define SIG_ERR ((sighandler_t)-1)
	#define SIG_IGN ((sighandler_t)1)
	
	#endif // _KERNEL_LIBKERN_BITS_SIGNAL_H
\end{lstlisting}





\newpage
\subsection{Estructuras de datos del planificador}

\begin{lstlisting}[
		language=C,
		label=prog:struct_runqueue,
		caption={Estruturas \texttt{runqueue} y \texttt{sched\_data}\\
		Archivo: \texttt{kernel/include/tasking/bits/sched.h}.}
	]
	struct runqueue {
		struct thread* head;
		struct thread* tail;
	};
	typedef struct runqueue runqueue_t;
	
	struct sched_data {
		int next_read_prio;
		runqueue_t* master_buf;
		runqueue_t* slave_buf;
		int enqueued_tasks;
	};
	typedef struct sched_data sched_data_t;
\end{lstlisting}

\begin{lstlisting}[
	language=C,
	label=prog:struct_trapframe_x86,
	caption={Estructura de datos del \textit{trapframe} para \texttt{x86}.\\
	Archivo: \texttt{kernel/include/platform/x86/i386/tasking/trapframe.h}.}
	]
	struct PACKED trapframe {
		// registers as pushed by pusha
		uint32_t edi;
		uint32_t esi;
		uint32_t ebp;
		uint32_t oesp; // useless & ignored
		uint32_t ebx;
		uint32_t edx;
		uint32_t ecx;
		uint32_t eax;
		
		// rest of trap frame
		uint16_t gs;
		uint16_t padding1;
		uint16_t fs;
		uint16_t padding2;
		uint16_t es;
		uint16_t padding3;
		uint16_t ds;
		uint16_t padding4;
		uint32_t int_no;
		
		// below here defined by x86 hardware
		uint32_t err;
		uint32_t eip;
		uint16_t cs;
		uint16_t padding5;
		uint32_t eflags;
		
		// below here only when crossing rings, such as from user to kernel
		uint32_t esp;
		uint16_t ss;
		uint16_t padding6;
	};
	typedef struct trapframe trapframe_t;
\end{lstlisting}

\begin{lstlisting}[
	language=C,
	label=prog:struct_trapframe_ARM,
	caption={Estructura de datos del \textit{trapframe} para \texttt{ARM}.\\
		Archivo: \texttt{kernel/include/platform/arm32/tasking/trapframe.h}.}
	]
	typedef struct {
		uint32_t user_flags;
		uint32_t user_sp;
		uint32_t user_lr;
		uint32_t r[13];
		uint32_t user_ip;
	} PACKED trapframe_t;
\end{lstlisting}


\begin{lstlisting}[
	language=C,
	label=prog:struct_context_x86,
	caption={Estructura de datos del \textit{context} para \texttt{x86}.\\
		Archivo: \texttt{kernel/include/platform/x86/i386/tasking/context.h}.}
	]
	struct PACKED context {
		uint32_t edi;
		uint32_t esi;
		uint32_t ebx;
		uint32_t ebp;
		uint32_t eip;
	};
	typedef struct context context_t;
\end{lstlisting}

\begin{lstlisting}[
	language=C,
	label=prog:struct_context_ARM,
	caption={Estructura de datos del \textit{context} para \texttt{ARM}.\\
		Archivo: \texttt{kernel/include/platform/arm32/tasking/context.h}.}
	]
	typedef struct {
		uint32_t r[9];
		uint32_t lr;
	} PACKED context_t;
\end{lstlisting}


\newpage
\subsection{Estructuras de datos del administrador de memoria}
\begin{lstlisting}[
	language=C,
	label=prog:struct_mapentry,
	caption={Estructura de datos de la tabla de mapeo de memoria.\\
		Archivo: \texttt{kernel/include/platform/generic/vmm/mapping\_table.h}.}
	]
	struct mapping_entry {
		uintptr_t paddr;
		uintptr_t vaddr;
		size_t pages;
		uint32_t flags;
		uint32_t last; // 1 if an element is the last.
	};
	typedef struct mapping_entry mapping_entry_t;
	
	extern mapping_entry_t extern_mapping_table[]; // Maps after kernel tables are ready, so can be outside kernelspace
\end{lstlisting}

\begin{lstlisting}[
	language=C,
	label=prog:struct_memzone,
	caption={Estructura de datos de la zona de memoria usada por un archivo.\\
		Archivo: \texttt{kernel/include/mem/memzone.h}.}
	]
	struct memzone {
		uintptr_t vaddr;
		size_t len;
		mmu_flags_t mmu_flags;
		uint32_t type;
		file_t* file;
		off_t file_offset;
		size_t file_size;
		struct vm_ops* ops;
	};
	typedef struct memzone memzone_t;
\end{lstlisting}

\begin{lstlisting}[
	language=C,
	label=prog:struct_memop,
	caption={Estructura de datos de las operaciones que se pueden realizar a la memoria.\\
		Archivo: \texttt{kernel/include/mem/vmm.h}.}
	]
	struct vm_ops {
		int (*load_page_content)(struct memzone* zone, uintptr_t vaddr);
		int (*swap_page_mode)(struct memzone* zone, uintptr_t vaddr);
		int (*restore_swapped_page)(struct memzone* zone, uintptr_t vaddr);
	};
	typedef struct vm_ops vm_ops_t;
\end{lstlisting}

\begin{lstlisting}[
	language=C,
	label=prog:struct_memvspace,
	caption={Estructura de datos del espacio virtual de la memoria.\\
		Archivo: \texttt{kernel/include/mem/vm\_address\_space.h}.}
	]
	struct vm_address_space {
		ptable_t* pdir;
		dynamic_array_t zones;
		int count;
		spinlock_t lock;
	};
	typedef struct vm_address_space vm_address_space_t;
\end{lstlisting}

\begin{lstlisting}[
	language=C,
	label=prog:struct_mempaget,
	caption={Estructura de datos de la tabla de páginas.\\
		Archivo: \texttt{kernel/include/mem/bits/vm.h}.}
	]
	typedef struct {
		ptable_entity_t entities[1];
	} ptable_t;
\end{lstlisting}


\newpage
\subsection{Estructuras de datos del sistema de archivos}
\begin{lstlisting}[
	language=C,
	label=prog:struct_files,
	caption={Definición de archivo y descriptor de archivos.\\
		Archivo: \texttt{kernel/include/fs/vfs.h}.}
	]
	struct file {
		size_t count;
		file_type_t type;
		union {
			dentry_t* dentry; // type == FTYPE_FILE
			struct socket* socket; // type == FTYPE_SOCKET
		};
		uint32_t flags;
		path_t path;
		file_ops_t* ops;
		
		// Used by socket to keep data.
		void* auxdata;
		
		// Protects flags.
		spinlock_t lock;
	};
	typedef struct file file_t;
	
	struct file_descriptor {
		file_t* file;
		off_t offset;
		int flags;
	};
	typedef struct file_descriptor file_descriptor_t;
\end{lstlisting}

\begin{lstlisting}[
	language=C,
	label=prog:struct_ext2,
	caption={Estructuras de datos del sistema de archivos \texttt{ext2} ligero.\\
		Archivo: \texttt{boot/libboot/fs/ext2\_lite.h}.}
	]
	typedef struct {
		uint32_t inodes_count;
		uint32_t blocks_count;
		uint32_t r_blocks_count;
		uint32_t free_blocks_count;
		uint32_t free_inodes_count;
		uint32_t first_data_block;
		uint32_t log_block_size;
		uint32_t log_frag_size;
		uint32_t blocks_per_group;
		uint32_t frags_per_group;
		uint32_t inodes_per_group;
		uint32_t mtime;
		uint32_t wtime;
		uint16_t mnt_count;
		uint16_t max_mnt_count;
		uint16_t magic;
		uint16_t state;
		uint16_t errors;
		uint16_t minor_rev_level;
		uint32_t lastcheck;
		uint32_t checkinterval;
		uint32_t creator_os;
		uint32_t rev_level;
		uint16_t def_resuid;
		uint16_t def_resgid;
		
		uint32_t first_ino;
		uint16_t inode_size;
		uint16_t block_group_nr;
		uint32_t feature_compat;
		uint32_t feature_incompat;
		uint32_t feature_ro_compat;
		uint8_t uuid[16];
		uint8_t volume_name[16];
		uint8_t last_mounted[64];
		uint32_t algo_bitmap;
		
		uint8_t prealloc_blocks;
		uint8_t prealloc_dir_blocks;
		
		// current jurnalling is unsupported
		uint8_t unused[1024 - 206];
	} superblock_t;

	typedef struct {
		uint32_t block_bitmap;
		uint32_t inode_bitmap;
		uint32_t inode_table;
		uint16_t free_blocks_count;
		uint16_t free_inodes_count;
		uint16_t used_dirs_count;
		uint16_t pad;
		uint8_t reserved[12];
	} group_desc_t;

	typedef struct {
		uint16_t mode;
		uint16_t uid;
		uint32_t size;
		uint32_t atime;
		uint32_t ctime;
		uint32_t mtime;
		uint32_t dtime;
		uint16_t gid;
		uint16_t links_count;
		uint32_t blocks;
		uint32_t flags;
		uint32_t osd1;
		uint32_t block[15];
		uint32_t generation;
		uint32_t file_acl;
		uint32_t dir_acl;
		uint32_t faddr;
		uint32_t osd2[3];
	} inode_t;

	typedef struct {
		uint32_t inode;
		uint16_t rec_len;
		uint8_t name_len;
		uint8_t file_type;
		char* name; // may be a problematic for 64bit versions
	} dir_entry_t;
\end{lstlisting}




\subsection{Estructuras de datos de los controladores de dispositivo}
\begin{lstlisting}[
	language=C,
	label=prog:struct_device,
	caption={Definición de dispositivos.\\
		Archivo: \texttt{kernel/include/drivers/bits/device.h}.}
	]
	struct device_desc_pci {
		uint8_t bus;
		uint8_t device;
		uint8_t function;
		uint16_t vendor_id;
		uint16_t device_id;
		uint8_t class_id;
		uint8_t subclass_id;
		uint8_t interface_id;
		uint8_t revision_id;
		uint32_t interrupt;
		uint32_t port_base;
	};
	typedef struct device_desc_pci device_desc_pci_t;
	
	struct device_desc {
		int type;
		union {
			device_desc_pci_t pci;
			device_desc_devtree_t devtree;
		};
		uint32_t args[4];
	};
	typedef struct device_desc device_desc_t;
	
	struct device {
		int id;
		int type;
		bool is_virtual;
		int driver_id;
		
		spinlock_t lock;
		device_desc_t device_desc;
	};
	typedef struct device device_t;
	
	
\end{lstlisting}

\begin{lstlisting}[
	language=C,
	label=prog:struct_devtree,
	caption={Definición del árbol de dispositivos.\\
		Archivo: \texttt{kernel/include/drivers/devtree.h}.}
	]
	
	struct PACKED devtree_header {
		char signature[8];
		uint32_t revision;
		uint32_t flags;
		uint32_t entries_count;
		uint32_t name_list_offset;
	};
	typedef struct devtree_header devtree_header_t;
	
	struct PACKED devtree_entry {
		uint32_t type;
		uint32_t flags;
		uint64_t region_base;
		uint64_t region_size;
		uint32_t irq_lane;
		uint32_t irq_flags;
		uint32_t irq_priority;
		uint32_t rel_name_offset;
		uint64_t aux1;
		uint64_t aux2;
		uint64_t aux3;
		uint64_t aux4;
	};
	typedef struct devtree_entry devtree_entry_t;
\end{lstlisting}




\newpage
\subsection{Algoritmo de arreglo dinámico}
\begin{lstlisting}[
	language=C,
	label=prog:struct_dynamic_array,
	caption={Estructura de datos del Algoritmo de arreglo dinámico.\\
		Archivo: \texttt{kernel/include/algo/dynamic\_array.h}.}
	]
	struct dynamic_array_bucket {
		void* data;
		struct dynamic_array_bucket* next;
		size_t capacity;
		size_t size;
	};
	typedef struct dynamic_array_bucket dynamic_array_bucket_t;
	
	struct dynamic_array {
		dynamic_array_bucket_t* head;
		dynamic_array_bucket_t* tail;
		size_t size; /* number of elements in vector */
		size_t element_size; /* size of elements in bytes */
	};
	typedef struct dynamic_array dynamic_array_t;
\end{lstlisting}